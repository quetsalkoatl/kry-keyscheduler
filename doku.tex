\documentclass[12pt, letterpaper]{article}

\usepackage[a4paper, total={6in, 8in}]{geometry}
\usepackage[utf8]{inputenc}
\usepackage{tabls}
\usepackage{xcolor}

\setlength{\parindent}{0pt}
\setlength{\parskip}{10pt}

\definecolor{dgreen}{RGB}{0, 192, 0}

\newcommand{\code}[1]{\texttt{#1}}

\title{AES Keyscheduler \\
\begin{large}
kry Projekt
\end{large} }
\author{Moritz Roth}
\date{4. April 2019}

\begin{document}

\maketitle

\section*{Einleitung}
Die Aufgabe war es, den Keyschedule-Algorithmus der AES Verschlüsselung für 128 bit Keys zu implementieren, so wie dieser im Standardisierungsdokument beschrieben ist.

Für die Implementation habe ich die Sprache Java gewählt, da ich mit dieser am besten vertraut bin. Ausserdem habe ich nach dem Keyscheduler noch versucht den kompletten Verschlüsselungsalgorithus zu implementieren, dafür hat mir allerdings dann die Zeit gefeht.

\section*{Keyschduler Implementation}

Das Projekt ist ein Maven-Projekt, da ich so im Code sehr einfach Testfälle definieren kann, die dann ausgeführt werden können. Der Implementationscode befindet sich im Verzeichnis \code{src/main/java/ch/fhnw/kry/keyscheduler}.


Für den Keyscheduler wichtig sind die Klassen \code{KeySchedule.java}, \code{SubBytes.java}, \code{Byte.java} und \code{Word.java}, sowie die Enums \code{AESConfig.java} und \code{Base.java}.

\begin{center}
  {\tablinesep=10pt\begin{tabular}{l|p{10cm}}
  Klasse/Enum & Beschreibung \\ \hline
  KeySchedule.java & Enthält den eigentlichen Keyexpansion Alorithmus, der den Key von 16 Byte auf 44 Byte expandiert. \\
  SubBytes.java & Enthält Methoden, die als Parameter ein Byte erwarten (in verschienden Basen) und das entsprechende SubByte aus der S-Box Tabelle zurückgibt. \\
  Byte.java & Ein Byte Objekt, das das Byte intern als Integer speichert. Es kann ein Byte aus dezimal-, binär- und hexadezimal-Zahlen erstellen und dieses auch in diesen Basen zurückgeben. \\
  Word.java & Enthält ein Array aus 4 Bytes. \\
  AESConfig.java & Dieses Enum enthält die Konfigurationen für die verschiedenen Keylängen. \\
  Base.java & Dieses Enum enthält Details zu den Basen HEX (hexadezimal), DEZ (dezimal) und BIN (binär).
  \end{tabular}}
\end{center}

Die Klasse \code{KeySchedule.java} ist die Hauptklasse für das Projekt, sie enthält die Methode \code{public static Word[] keyExpansion(Byte[] key, AESConfig conf)}. Dieser kann man einen Key in form eines Bytearrays mitgeben, sowie eine AESConfig (also AES\_128, AES\_192 oder AES\_256) und sie gibt ein Wordarray zurück, das den expanded Key enthält. Wobei AES\_192 und AES\_256 noch nicht funktionieren, weil das round constant Array noch nicht dafür angepasst ist.

Bei der Implementation habe ich mich sehr genau an den Pseudocode aus der Präsentation gehalten.

\newpage
\section*{Tests}
Die Tests für den Keyscheduler befinden sich in der Testklasse \code{src/test/java/ch/fhnw /kry/keyscheduler/KeyScheduleTest.java}. Die Testfälle habe ich von der Webseite https://www.samiam.org/key-schedule.html. Es gibt jeweils ein Test für die edge cases (alles \code{00} und alles \code{ff}), einen bei dem die 16 Bytes im Key durchgezähhlt sind (\code{00} bis \code{0f}) und einen Key dazwischen, ausserdem habe ich einen Test erstellt, der fehlschlägt, weil ein Bit in der Ausgabe verändert wurde und zwei Tests, die falsche Schlüssel übergeben.

\subsection*{\code{KeyScheduleTest.java}}

\code{testKeyExpansion1()}:

INPUT\\
\code{00 00 00 00 00 00 00 00 00 00 00 00 00 00 00 00}

EXPECTED OUTPUT\\
\code{
00 00 00 00 00 00 00 00 00 00 00 00 00 00 00 00\\
62 63 63 63 62 63 63 63 62 63 63 63 62 63 63 63\\
9b 98 98 c9 f9 fb fb aa 9b 98 98 c9 f9 fb fb aa\\
90 97 34 50 69 6c cf fa f2 f4 57 33 0b 0f ac 99\\
ee 06 da 7b 87 6a 15 81 75 9e 42 b2 7e 91 ee 2b\\
7f 2e 2b 88 f8 44 3e 09 8d da 7c bb f3 4b 92 90\\
ec 61 4b 85 14 25 75 8c 99 ff 09 37 6a b4 9b a7\\
21 75 17 87 35 50 62 0b ac af 6b 3c c6 1b f0 9b\\
0e f9 03 33 3b a9 61 38 97 06 0a 04 51 1d fa 9f\\
b1 d4 d8 e2 8a 7d b9 da 1d 7b b3 de 4c 66 49 41\\
b4 ef 5b cb 3e 92 e2 11 23 e9 51 cf 6f 8f 18 8e
}

RESULTAT: {\color{dgreen}Erfolgreich}

\newpage
\code{testKeyExpansion2()}:

INPUT\\
\code{ff ff ff ff ff ff ff ff ff ff ff ff ff ff ff ff}

EXPECTED OUTPUT\\
\code{
ff ff ff ff ff ff ff ff ff ff ff ff ff ff ff ff\\
e8 e9 e9 e9 17 16 16 16 e8 e9 e9 e9 17 16 16 16\\
ad ae ae 19 ba b8 b8 0f 52 51 51 e6 45 47 47 f0\\
09 0e 22 77 b3 b6 9a 78 e1 e7 cb 9e a4 a0 8c 6e\\
e1 6a bd 3e 52 dc 27 46 b3 3b ec d8 17 9b 60 b6\\
e5 ba f3 ce b7 66 d4 88 04 5d 38 50 13 c6 58 e6\\
71 d0 7d b3 c6 b6 a9 3b c2 eb 91 6b d1 2d c9 8d\\
e9 0d 20 8d 2f bb 89 b6 ed 50 18 dd 3c 7d d1 50\\
96 33 73 66 b9 88 fa d0 54 d8 e2 0d 68 a5 33 5d\\
8b f0 3f 23 32 78 c5 f3 66 a0 27 fe 0e 05 14 a3\\
d6 0a 35 88 e4 72 f0 7b 82 d2 d7 85 8c d7 c3 26
}

RESULTAT: {\color{dgreen}Erfolgreich}

\code{testKeyExpansion3()}:

INPUT\\
\code{00 01 02 03 04 05 06 07 08 09 0a 0b 0c 0d 0e 0f}

EXPECTED OUTPUT\\
\code{
00 01 02 03 04 05 06 07 08 09 0a 0b 0c 0d 0e 0f\\
d6 aa 74 fd d2 af 72 fa da a6 78 f1 d6 ab 76 fe\\
b6 92 cf 0b 64 3d bd f1 be 9b c5 00 68 30 b3 fe\\
b6 ff 74 4e d2 c2 c9 bf 6c 59 0c bf 04 69 bf 41\\
47 f7 f7 bc 95 35 3e 03 f9 6c 32 bc fd 05 8d fd\\
3c aa a3 e8 a9 9f 9d eb 50 f3 af 57 ad f6 22 aa\\
5e 39 0f 7d f7 a6 92 96 a7 55 3d c1 0a a3 1f 6b\\
14 f9 70 1a e3 5f e2 8c 44 0a df 4d 4e a9 c0 26\\
47 43 87 35 a4 1c 65 b9 e0 16 ba f4 ae bf 7a d2\\
54 99 32 d1 f0 85 57 68 10 93 ed 9c be 2c 97 4e\\
13 11 1d 7f e3 94 4a 17 f3 07 a7 8b 4d 2b 30 c5
}

RESULTAT: {\color{dgreen}Erfolgreich}

\newpage
\code{testKeyExpansion4()}:

INPUT\\
\code{69 20 e2 99 a5 20 2a 6d 65 6e 63 68 69 74 6f 2a}

EXPECTED OUTPUT\\
\code{
69 20 e2 99 a5 20 2a 6d 65 6e 63 68 69 74 6f 2a\\
fa 88 07 60 5f a8 2d 0d 3a c6 4e 65 53 b2 21 4f\\
cf 75 83 8d 90 dd ae 80 aa 1b e0 e5 f9 a9 c1 aa\\
18 0d 2f 14 88 d0 81 94 22 cb 61 71 db 62 a0 db\\
ba ed 96 ad 32 3d 17 39 10 f6 76 48 cb 94 d6 93\\
88 1b 4a b2 ba 26 5d 8b aa d0 2b c3 61 44 fd 50\\
b3 4f 19 5d 09 69 44 d6 a3 b9 6f 15 c2 fd 92 45\\
a7 00 77 78 ae 69 33 ae 0d d0 5c bb cf 2d ce fe\\
ff 8b cc f2 51 e2 ff 5c 5c 32 a3 e7 93 1f 6d 19\\
24 b7 18 2e 75 55 e7 72 29 67 44 95 ba 78 29 8c\\
ae 12 7c da db 47 9b a8 f2 20 df 3d 48 58 f6 b1
}

RESULTAT: {\color{dgreen}Erfolgreich}

\code{testKeyExpansionFail()}:

In diesem Test ist der EXPECTED OUTPUT falsch (die rote \code{{\color{red}9}} in der letzten Zeile). Dieser Test soll zeigen, dass das erwartete Resultat auch wirklich mit dem tatsächlichen Output verglichen wird.

INPUT\\
\code{69 20 e2 99 a5 20 2a 6d 65 6e 63 68 69 74 6f 2a}

EXPECTED OUTPUT\\
\code{
69 20 e2 99 a5 20 2a 6d 65 6e 63 68 69 74 6f 2a\\
fa 88 07 60 5f a8 2d 0d 3a c6 4e 65 53 b2 21 4f\\
cf 75 83 8d 90 dd ae 80 aa 1b e0 e5 f9 a9 c1 aa\\
18 0d 2f 14 88 d0 81 94 22 cb 61 71 db 62 a0 db\\
ba ed 96 ad 32 3d 17 39 10 f6 76 48 cb 94 d6 93\\
88 1b 4a b2 ba 26 5d 8b aa d0 2b c3 61 44 fd 50\\
b3 4f 19 5d 09 69 44 d6 a3 b9 6f 15 c2 fd 92 45\\
a7 00 77 78 ae 69 33 ae 0d d0 5c bb cf 2d ce fe\\
ff 8b cc f2 51 e2 ff 5c 5c 32 a3 e7 93 1f 6d 19\\
24 b7 18 2e 75 55 e7 72 29 67 44 95 ba 78 29 8c\\
ae 12 7c da db 47 9{\color{red}9} a8 f2 20 df 3d 48 58 f6 b1
}

RESULTAT: {\color{dgreen}Erfolgreich}\\
$\rightarrow$ heisst, dass der Vergleich zwischen tatsächlichem und erwartetem Output \code{false} zurückgibt. Darum auch \code{assertFalse()}.

\code{testKeyExpansionException1()} / \code{testKeyExpansionException2()}:

Diese beiden Tests failen mit einer \code{IllegalArgumentException}, weil der Key zu lange bzw. zu kurz ist. Die Tests sind {\color{dgreen}erfolgerich} im Sinne, dass die Exception geworfen wird.

Weitere Tests:

Ansonsten habe ich noch einige Tests für Byte, SubBytes und Word geschrieben, um zu testen, ob sie sich wie erwartet verhalten. Dazu gehören auch Tests von \code{xor()} und \code{rotWord()}. Diese sind alle {\color{dgreen}erfolgerich}.

\newpage
\section*{Weitere Implementation AES}
Nach der Implementation habe ich mich noch an der Implementation weiterer Teile des AES versucht. Die Verschlüsselung habe ich auch beendet, die Entschlüsselung allerdings nicht. In der Klasse \code{AES.java} befinden sich die Methoden \code{cypher()} für die Verschlüsseung und \code{eqInvCipher()} für die Entschlüsselung. Man kann jeweils den Plaintext als Bytearray und den expanded Key als Wordarray mitgeben, sowie die Schlüssellänge (die könnte auch anhand des Keys berechnet werden).

Das rcon-Array in der Keyexpansion müsste noch for die grösseren Keysizes erweitert (oder generiert) werden.

Die anderen Klassen sind jeweils selbstredend: \code{AddRoundKey.java} enthält die addRoundKey Methode (und Inverse), \code{MixColumns.java} enthält die mixColumns Methode (und Inverse) und \code{ShiftRows.java} enthält die shiftRows Methode (und Inverse). 

Zudem habe ich in der Klasse \code{MixColumns.java} noch die Galois-Multiplikation implementiert (\code{gmul()}).

$\rightarrow$ Alle diese Klassen und Methoden sind allerding ungetestet!

\end{document}

